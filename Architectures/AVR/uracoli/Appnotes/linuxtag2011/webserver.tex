% $Id$
\section{Webserver}
%
%
"Uber die serielle Schnittstelle des Host-PCs werden nun die Daten der Sensorknoten empfangen
und es k"onnen auch Daten an diese gesendet werden. 
Als einfaches Beispiel wird hier das Auslesen und
Darstellen der von den Sensorknoten gemessenen Temperaturen mit Hilfe 
eines Webservers gezeigt. Dieser ist in Python geschrieben und verwendet
die Module \texttt{Simple\-HTTP\-Server} und \texttt{SocketServer}.
Das Server wird mit dem Befehl \texttt{python webserver.py} gestartet und 
ist danach unter \texttt{http://localhost:8080} erreichbar.

\begin{lstlisting}[tabsize=4]
import SocketServer
import SimpleHTTPServer
...
class CustomHandler(SimpleHTTPServer.SimpleHTTPRequestHandler):
    def do_GET(self):
        global sdev
        self.send_response(200)
        self.send_header('Content-type','text/html')
        self.end_headers()
        txt = sdev.build_sensor_html_table()
        self.wfile.write(header() + txt + footer())

if __name__ == "__main__":
	...
    httpd = SocketServer.ThreadingTCPServer(
                    ('localhost', PORT),CustomHandler)
	...
	httpd.serve_forever()
\end{lstlisting}

In einem separaten Thread werden die ankommenden Sensordaten von der
seriellen Schnittstelle gelesen und in einem Dictionary gespeichert.
Die Funktion {\tt build\_sensor\_html\_table()} baut daraus eine HTML-Tabelle. 
Diese Methode wird von der oben beschriebenen Serverfunktion 
{\tt do\_GET()} aus aufgerufen.
\begin{lstlisting}[tabsize=4]
class Device:
    def __init__(self,port, baud, verbose=0):
        self.sport = serial.Serial(port, baud)
        self.rxThread = threading.Thread(target = self._rx_thread_)
        ...
        self.rxThread.start()
        self.sensor_data = {} # collects sensor data

    # function running in thread
    def _rx_thread_(self):
        while 1:
            sline = self.sport.readline().strip()
            tim = int(time.time())
            (node, temp) = self.parse_line(sline)
            self.sensor_data[node] = (tim, temp)

    def build_sensor_html_table(self):
        ...
        for node in nodes:
            ...
            (tim,temp) = self.sensor_data[node]
            if (int(time.time()) - tim) > 10:
                ret += '<td style="color:red">'\
                       '%s</td>\n' % temp
            else:  
                ret += '<td>%s</td>\n' % temp
        ...
        return ret
\end{lstlisting}

